\section{Introduction}
\label{sec:introduction}

In many ways, the kitchen can be seen as the world's oldest laboratory.
People have experimented with a large number of ingredients, their complex interplay and preparation methods ever since we start using tools.
With the advent of Web 2.0 and its focus on user-generated content and social media, a lot of data about cooking is now available.
On websites such as Allrecipes and Food Network, users can submit their own recipes and rate recipes of other users.
Instead of browsing through cookbooks, food lovers can now discover new recipes by limiting themselves to a specific category, such as `Christmas' or `Quick \& Easy', or by following other users on the platform.

While these are interesting developments for cooks, it also allows us to study the recipes in much greater detail.
Traditionally, clustering and classification methods have been used to identify consumer patterns, while more recently, matrix factorization methods have been applied.
It is often assumed that if a recipe can be found with a certain pairing of ingredients, this pairing must be generally favorable.
Using the user ratings available, however, it becomes possible to find more specific patterns for certain users groups.
For example, the Food Pairing theory states that ingredients with overlapping flavor compounds are often used in Western cuisine.
It is often thought that this theory does not hold for Eastern cuisine, which might indicate a strong cultural bias towards certain combinations of ingredients in relation to the combination of flavor compounds.
Now that a large volume of recipes and user reviews is available, we can start to find out what ingredients make a good combination in much finer detail.

In this research, several aspects of recipes and ingredient pairing are studied using a 