\section{Related Work}
\label{sec:related_work}

Using data science to better understand the fundamental theories that underly our perception of food is a relatively new effort.
With the recent advent of social networks focussing on food and the availability of chemical information about food products, it is now possible to analyze much larger quantities of data.
While chemical engineering has since long influenced the search for flavor enhancing additives, little is known about what makes a good recipe.
One well-known, but still controversial, theory is that of Food Pairing\textsuperscript{\textregistered}, which states that ingredients with overlapping flavors make a good combination \citep{FoodPairing}.
Although some research has shown the theory to hold for Western cuisine, it does not for the Eastern and southern European cuisines \citep{Ahn2011}\citep{Jain2015}.
Others have stated that the hypothesis lacks a solid basis and even more, simpler food pairing alternatives exist \citep{Klepper2011}.
What is still unargued is that aroma, the sensory perception of chemicals, makes up for about $80$ percent of our eating experience, making the chemical composition of foodstuffs an excellent subject of study.
Even though many more factors influence our experience, such as texture, temperature and sound, the ingredients themselves are a natural starting point in the analysis of recipes.
One interesting direction in this way is that of \emph{recipe completion}.
This is an information retrieval task in which the goal is to complete a recipe, given its ingredient list \citep{Clercq2016}.
The use of \emph{Non-negative Matrix Factorization} (NMF) is a promising approach that is both able to retrieve the missing ingredients and creates an understandable model of a recipe database \citep{Zetlaoui2011}\citep{Clercq2016}.