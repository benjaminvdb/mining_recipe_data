\section{Related Work}
\label{sec:related_work}

Utilizing data science to study the fundamental theories that underly our perception of food is a relatively new effort.
With the recent introduction of social networks focussing on food and the availability of chemical information about food products \citep{FooDB}, it is now possible to analyze much larger quantities of data.
While chemical engineering has since long influenced the search for flavor enhancing additives, little is known about what makes a good recipe.
One well-known, but still controversial, theory is that of Food Pairing\textsuperscript{\textregistered}, which states that ingredients with overlapping flavors make a good combination \citep{FoodPairing}.
Although some research has shown the theory to hold for Western cuisine, it does not for the Eastern and southern European cuisines \citep{Ahn2011, Jain2015}.
Others have stated that the hypothesis lacks a solid basis and even more, simpler food pairing alternatives exist \citep{Klepper2011}.
What is still unargued is that an interesting combination of ingredients is one of the most important components of a successful recipe, making the ingredient lists themselves a good starting point for study.
One interesting direction in this way is that of \emph{recipe completion}, an information retrieval task in which the goal is to complete a recipe, given its ingredient list \citep{Clercq2016}.
The use of \emph{Non-negative Matrix Factorization} (NMF) is a promising approach that is both able to retrieve the missing ingredients and creates an understandable model of a recipe database \citep{Zetlaoui2011, Clercq2016}.
In this study we try two other methods for recipe completion using association rules and item similarities.